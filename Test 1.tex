
\documentclass{article}
\usepackage{amsmath}
\usepackage{setspace}
\usepackage{geometry}
\geometry{legalpaper, portrait, margin=1in}
\author{Artem Lebedev}
\title{W203, Test 1}
\date{\today}


\begin{document}

\maketitle

\begin{enumerate}
\item[Q 2.1] Solve for the constant\\
	Given that
	\begin{equation}
		\int_{-\infty}^{+\infty} \int_{-\infty}^{+\infty}f_{X,Y}(x,y) \,dx\,dy = 1
	\end{equation}
	we can write for this case:
	\begin{equation}
		\int_{x=0}^{x=2} \int_{y=0}^{y=2}c\,*x^2\,y \,dx\,dy = 1
	\end{equation}
	consequently:
	\begin{equation}
		c*\Big|_{x=0}^{x=2} \frac{x^3}{3}\Big|_{y=0}^{y=2}\frac{y^2}{2} = 
		\frac{c}{6}\,*2^3\,*2^2 = 1
	\end{equation}
	Therefore $$c = \frac{3}{16}$$
\item[Q 2.2] Probability a la mode\\
	We can derive PDF for Y from joint PDF of X and Y by finding $f_Y(y)$, marginal PDF for Y:
	\begin{equation}
		f_Y(y) = \int_{0}^{2}f_{X,Y}(x,y)\,dx
	\end{equation}
	\begin{align*}
		f_Y(y) = \int_{0}^{2}\frac{3}{16}\,x^2\,y\,dx=\frac{3}{16}*\Big|_{x=0}^{x=2}\frac{x^3*y}{3}=\\
		= 	\begin{cases}
				\frac{y}{2}\;0\leq y \leq2\\
				0,\; elsewhere\\
			\end{cases}
	\end{align*}
	Since this is a continuous differnetiable function on the interval $[0;2]$, it should have maxiumum on this interval (Extreme value theorem). Since this is an increasing linear function, it will have its maximum at the maximum of its argument, i.e $$f_{Y_{max}}(y) = \frac{2}{2} = 1$$
\item[Q 2.3] Expectation of Y, E[Y]\\
	By definition, expectation of a continuous random variable:
	\begin{equation}
		E[Y] = \int_{-\infty}^{+\infty}y*f_Y(y)\,dy
	\end{equation}
	applying this definition to our case, using Eq. 4:
	\begin{equation}
		E[Y] = \int_{0}^{2}y*\frac{y}{2}\,dy = \Big|_0^2\frac{y^3}{2*3} = \frac{4}{3}
	\end{equation}
\item[Q 2.4] Variance of Y, V[Y]\\
	Using a simple modification of the definition for variance of a continuous random variable, and applying LOTUS and result 6:
	\begin{equation}
		V[Y] = E[y^2]-E^2[y] = \int_{0}^{2}y^2\,\frac{y}{2}\,dy - \left(\frac{4}{3}\right)^2 = \Big|_0^2\left[\frac{y^4}{4*2}\right] - \frac{16}{9} = \frac{8}{31}
	\end{equation}

\end{enumerate}

\end{document}